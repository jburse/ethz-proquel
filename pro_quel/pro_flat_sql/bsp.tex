\documentstyle[german]{report}
\setlanguage{\german}

\topmargin 2cm
\oddsidemargin 1cm
\evensidemargin 1cm
\textwidth 15cm
\textheight 18cm

\begin{document}

\parindent0.0cm
\parskip2ex
\topsep0.0cm 
\partopsep0.0cm 
\itemsep0.0cm 
\parsep0.0cm

\abovedisplayskip0.0cm
\belowdisplayskip0.0cm
\abovedisplayshortskip0.0cm
\belowdisplayshortskip0.0cm

\chapter{Einleitung}

bla bla \\
bla bla \\
bla bla

\section{Motivation}

Semantische Query Optimierung soll im folgenden anhand eines kleinen Beispieles
illustriert werden. Das Beispiel ist aus [King] ... baut auf folgendem 
Datenbankschema auf:

\begin{tabular}{ll}
\quad & ships(SHIP,STYPE) \\
\quad & cargoes(SHIP,DEST,CTYPE,QUAN)
\end{tabular}

\begin{figure}
\begin{center}
\begin{tabular}{l|l}
SHIP & STYPE \\        
\hline 
bralanta & tanker \\            
british Wye & bulk \\               
carlova & pressurized tanker 
\end{tabular}
\end{center}
\caption{Die $ships$ Relation}
\label{tabships}
\end{figure}

\begin{figure}
\begin{center}
\begin{tabular}{l|l|l|l}
SHIP & DEST & CTYPE & QUAN \\
\hline
bralanta & hamburg & coffee & 1000 \\
bralanta & hamburg & bananas & 500 \\
carlova & marseille & liquefied nat. gas & 1500 \\
british wye & marseille & coffee & 1000 \\
british wye & marseille & bananas & 500
\end{tabular}
\end{center}
\caption{Die $cargoes$ Relation}
\label{tabcargoes}
\end{figure}

Die in unserem Beispiel verwendeten Daten sind in abb. \ref{tabships} und
\ref{tabcargoes} zu sehen. Ein Datenbanksystem erlaubt nun vielf"alltige 
Anfragen auf diese Daten. Im speziellen haben Anfragen einer deduktiven
Datenbank die Form von Formeln. So wird z.B. folgende Anfrage:

\begin{center}
\begin{minipage}{10cm}
`Zeige alle Schiffe und deren Ladungsmenge und Ladungstyp die
nach Marseille fahren und mindestens 1000 T geladen haben'
\end{minipage}
\end{center}

Durch folgende Formel ausgedr"uckt:

\begin{center}
$cargoes(SHIP,'Marseille',CTYPE,QUAN) \wedge QUAN>=1000$
\end{center}

Die Antwort lautet dann:

\begin{center}
\begin{tabular}{l|l|l}
SHIP & CTYPE & QUAN \\
\hline
carlova & liquefied nat. gas & 1500 \\
british wye & coffee & 1000
\end{tabular}
\end{center}

Die zuverl"assigkeit der Daten kann durch Konsistenzbedingungen erh"oht werden.
Konsistenzbedingungen zeigen logische Zusammanh"ange zwischen Relationen auf die Invariant
bleiben m"ussen. In deduktiven Datenbank sind Konsistenzbedingungen Formeln die immer erf"ullt 
sein m"ussen. So wird folgend Konsistenzbedingung:

\begin{center}
\begin{minipage}{10cm}
`Gas wird immer mit Drucktankern transportiert'
\end{minipage}
\end{center}

Druch folgende Formel ausgedr"uckt:

\begin{center}
$cargoes(SHIP,DEST,'liqu. nat. gas',QUAN) \rightarrow ships(SHIP,'pres. tank')$
\end{center}

Konsistenzbedingungen k"onnen nun zur Optimierung von Anfragen verwendet werden. Dazu betrachten
wir folgende Anfrage:

\begin{center}
\begin{minipage}{10cm}
`Zeige alle Drucktanker Schiffe und deren Ladungsmenge die Gas geladen haben
und nach Marseille fahren'
\end{minipage}
\end{center}

Die entsprechende Formel lautet:

\begin{center}
$ships(SHIP,'pres. tank') \wedge cargoes(SHIP,'Marseille','liqu. nat. gas',QUAN)$
\end{center}

Da Gas immer durch Drucktanker transportiert wird er"ubrigt sich die erste Bedingung. Wir
k"onnen also folgende vereinfachte Anfrage stellen und erhalten genau die gleiche Antwortmenge:

\begin{center}
$cargoes(SHIP,'Marseille','liqu. nat. gas',QUAN)$
\end{center}

Im gegensatz zur normalen Query Optimierung wird bei der Semantischen Query Optimierung 
zus"atzliche Semantische Information in der Form von Konsistenzbedingungen verwendet um eine 
einfachere Query zu erhalten. In diesem Beispiel konnte ein Join eleminiert werden und ...
In dieser Arbeit wurde ein allgemeines Verfahren zur Erzeugung der Kandidaten f"ur die 
modifizierte Query entwickelt.

\section{Problemstellung}

\end{document}
